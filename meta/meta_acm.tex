% #### Title and Authors ####
\title{KernelHaven---An Open Infrastructure for Product Line Analysis}
\author{Christian Kröher, Sascha El-Sharkawy, Klaus Schmid}
\affiliation{
  \institution{University of Hildesheim, Institute of Computer Science}
  \streetaddress{Universit{\"a}tsplatz 1}
  \city{Hildesheim}
	\country{Germany}
  \postcode{31134}
}
\email{{kroeher, elscha, schmid}@sse.uni-hildesheim.de}
% #### Title and Authors ####


% #### Abstract ####
\begin{abstract}
KernelHaven is an open infrastructure for Software Product Line (SPL) analysis. It is intended both as a production-quality analysis tool set as well as a research support tool, e.g., to support researchers in systematically exploring research hypothesis. For flexibility and ease of experimentation KernelHaven components are plug-ins for extracting certain information from SPL artifacts and processing this information, e.g., to check the correctness and consistency of variability information or to apply metrics. A configuration-based setup along with  automatic documentation functionality allows different experiments and supports their easy reproduction.

Here, we describe KernelHaven as a product line analysis research tool and highlight its basic approach as well as its fundamental capabilities. In particular, we describe available information extraction and processing plug-ins and how to combine them. 
On this basis, researchers and interested professional users can rapidly conduct a first set of experiments. Further, we describe the concepts for extending KernelHaven by new plug-ins, which reduces development effort when realizing new experiments.
%\ToDo{ Change the title?}
\end{abstract}
% #### Abstract ####


% #### Concepts and Keywords ####
%
% The code below should be generated by the tool at
% http://dl.acm.org/ccs.cfm
% Please copy and paste the code instead of the example below. 
%
\begin{CCSXML}
<ccs2012>
<concept>
<concept_id>10002944.10011123.10010912</concept_id>
<concept_desc>General and reference~Empirical studies</concept_desc>
<concept_significance>500</concept_significance>
</concept>
<concept>
<concept_id>10002944.10011123.10011131</concept_id>
<concept_desc>General and reference~Experimentation</concept_desc>
<concept_significance>500</concept_significance>
</concept>
<concept>
<concept_id>10011007.10011074.10011092.10011096.10011097</concept_id>
<concept_desc>Software and its engineering~Software product lines</concept_desc>
<concept_significance>500</concept_significance>
</concept>
</ccs2012>
\end{CCSXML}
\ccsdesc[500]{General and reference~Empirical studies}
\ccsdesc[500]{General and reference~Experimentation}
\ccsdesc[500]{Software and its engineering~Software product lines}
\keywords{Software product line analysis, variability extraction, static analysis, empirical software engineering}
% #### Concepts and Keywords ####


% ### Copyright ###
\copyrightyear{2018}
\acmYear{2018}
\setcopyright{rightsretained}
\acmConference[SPLC '18]{22nd International Systems and Software
Product Line Conference - Volume B}{September 10--14,
2018}{Gothenburg, Sweden}
\acmBooktitle{22nd International Systems and Software Product Line
Conference - Volume B (SPLC '18), September 10--14, 2018, Gothenburg,
Sweden}\acmDOI{10.1145/3236405.3236410}
\acmISBN{978-1-4503-5945-0/18/09}
% ### Copyright ###


\maketitle